
\documentclass{article}
\usepackage{amsmath}
\usepackage{graphicx} % Required for inserting images

\title{Comando “Time”}
\author{José M. Díaz M. (25682785)}
\date{19 de junio de 2025}

\begin{document}

\maketitle

\section{Descripción}

\quad

\textit{El comando “Time” es una herramienta fundamental en Unix/Linux que permite medir el tiempo de ejecución de un proceso, así como los recursos utilizados por un programa o comando. Proporciona tres mediciones distintas que permiten entender diferentes aspectos del rendimiento:}

\quad

\section{Tipos de tiempo y sus características}

\quad

\textbf{Tiempo Real (Real Time):}

\quad

\textbf{a)}{ El tiempo transcurrido desde el inicio hasta el final de la ejecución.}

\textbf{b)}{ Incluye cualquier tiempo de espera por recursos o operaciones de entrada/salida.}

\textbf{c)}{ Representa el tiempo percibido por el usuario}

\quad

\textbf{Tiempo de Usuario (User Time):}

\quad

\textbf{a)}{ Cantidad de Tiempo dedicado a ejecutar instrucciones en modo usuario.}

\textbf{b)}{ Solo cuenta el tiempo dedicado a ejecutar el código del programa.}

\textbf{c)}{ No incluye llamadas al sistema ni operaciones de entrada/salida.}

\quad

\textbf{Tiempo de Sistema (System Time):}

\quad

\textbf{a)}{ Calcula el tiempo de CPU empleado en modo kernel.}

\textbf{b)}{ Refleja el tiempo usado en llamadas al sistema operativo y operaciones privilegiadas.}

\textbf{c)}{ Indica la carga sobre el sistema operativo.}

\quad

\section{Ejemplo}


\begin{figure}[h]
    \centering
    \includegraphics[width=1\textwidth]{imgen_01.png}
    \caption{Ejemplo en lenguaje C}
    \label{fig:img1}
\end{figure}


\end{document}

